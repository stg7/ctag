\documentclass[
    ngerman, 
    a4paper, 
    headsepline,
    footsepline,
    12pt,
    fleqn    
    ]{scrartcl}

% packages
\usepackage[utf8]{inputenc}    
% \usepackage{times} %veraltet

% fülltext spielerein
\usepackage{lipsum}
\usepackage{blindtext} 

\usepackage{ulem}
\usepackage{fancybox}

\usepackage{graphicx,color}
\usepackage{url}

%\usepackage{mathrsfs}
%\usepackage{subfig}

\usepackage{listings}

\usepackage{longtable}
\usepackage{tabularx}       % Fuer Tabellen laenger als eine Seite
\usepackage{multicol,multirow}
\usepackage{colortbl} % farbige tabellen

\usepackage{latexsym}
\usepackage{amsmath}
\usepackage{float,placeins}

\renewcommand{\labelitemi}{$\triangleright$}
\renewcommand{\labelitemii}{$\diamond$}
\renewcommand{\labelitemiii}{$\circ$}

% Definitonen
% begin{definition} blabla bla \end{defitiniton} 
\newtheorem{definition}{Definition}


\usepackage{amssymb}


\usepackage[ngerman]{babel}
\usepackage[automark]{scrpage2}

\usepackage{qtree} % für Bäume 


% latex pictures
\usepackage{tikz}
\usetikzlibrary{shapes,snakes,arrows,matrix,calc,er}
%\usetikzlibrary{decorations,markings}


% Index erzeugen
%\usepackage{scrindex} \usepackage{makeidx}
%\makeindex{}

\usepackage{csquotes} % deutsche Anführungszeichen durch "` und  "'

%\usepackage[colorinlistoftodos,disable]{todonotes} % todonotes disable
\usepackage[colorinlistoftodos]{todonotes} % todonotes 

\usepackage{caption}
\usepackage{subcaption}

\usepackage{packages/wrapfig}

% farbdefinitionen
\definecolor{lightblue}{rgb}{0,0.6,1.0}
\definecolor{lightgray}{rgb}{0.2,0.2,0.2}
\definecolor{blue}{rgb}{0,0.3,0.6}
\definecolor{darkblue}{rgb}{0,0.1,0.4}
\definecolor{green}{rgb}{0,0.7,0.2}


% Paket für Links innerhalb des PDF Dokuments
\usepackage[%
	%pdftitle={\thema{}},% Titel der erzeugten PDF Datei
	%pdfauthor={\author{}},%
	%pdfcreator={LaTeX Skript by stg7},
	%pdfsubject={\thema{}},%
	%pdfkeywords={\schluesselwoerter{}}
	bookmarksopen=true,%
	bookmarksopenlevel=1,%
	plainpages=false%
	]{hyperref}
    
\hypersetup{%
	colorlinks=true,% links einfaerben, oder box drum malen?
	linkcolor={darkblue},% verweise im doc, ua inhaltsverzeichnis
	citecolor={blue},
	filecolor={black},
    filecolor=blue,
	urlcolor={green}
    }
    
% Kopf- und Fußzeile
\clearscrheadfoot 
\chead{\headmark} % automatischen Kapitelnamen rein
\ofoot[\pagemark]{\pagemark} % oben rechts Seitenzahl laut Richtlinie

\pagestyle{scrheadings}

% --------------------

% Seitendefinitionen

\setlength{\topmargin}{1.5cm}
\setlength{\headheight}{12pt}
\setlength{\headsep}{20pt}
\setlength{\topskip}{12pt}
\setlength{\evensidemargin}{0pt}
\setlength{\oddsidemargin}{0pt}
\setlength{\textheight}{240mm}
\setlength{\textwidth}{160mm}
\setlength{\voffset}{-2cm}
\setlength{\parindent}{0pt}
\setlength{\parskip}{6pt}


% Schrifteinstellungen
\usepackage[T1]{fontenc} 
\usepackage{lmodern} % moderne schriftart

\renewcommand*\familydefault{\sfdefault}

\setkomafont{sectioning}{\normalfont\bfseries}
\setkomafont{descriptionlabel}{\normalfont\bfseries}
\setkomafont{captionlabel}{\bfseries\footnotesize}
\setkomafont{caption}{\footnotesize}

% für dictum
\renewcommand*{\dictumwidth}{.5\textwidth}
\renewcommand*{\dictumauthorformat}[1]{\textsc{#1}\bigskip}


% eigene commands
\newcommand{\onecm}{\hspace{1cm}}
\newcommand{\hide}[1]{}

\newcommand{\whide}[1]{\textcolor{white}{#1}}

% Abbildungsreferenz
\newcommand{\picref}[1]{Abbildung~\ref{#1}}

% eventuelles Umbenennen des Literaturverzeichnisses in Quellen[verzeichnis]
%\renewcommand{\bibname}{Quellen} 

% rausblenden
\usepackage{ifthen}
\newboolean{includethis}

\newcommand{\zusatz}{}
\setboolean{includethis}{false} % abgabe version
%\setboolean{includethis}{true} % mit allem
\newcommand{\ifinclude}[1]{\ifthenelse{\boolean{includethis}}{#1}{}} 

\usepackage{setspace}
%su\onehalfspacing

\usepackage{enumitem}
